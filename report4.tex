\PassOptionsToPackage{unicode=true}{hyperref} % options for packages loaded elsewhere
\PassOptionsToPackage{hyphens}{url}
%
\documentclass[]{article}
\usepackage{lmodern}
\usepackage{amssymb,amsmath}
\usepackage{ifxetex,ifluatex}
\usepackage{fixltx2e} % provides \textsubscript
\ifnum 0\ifxetex 1\fi\ifluatex 1\fi=0 % if pdftex
  \usepackage[T1]{fontenc}
  \usepackage[utf8]{inputenc}
  \usepackage{textcomp} % provides euro and other symbols
\else % if luatex or xelatex
  \usepackage{unicode-math}
  \defaultfontfeatures{Ligatures=TeX,Scale=MatchLowercase}
\fi
% use upquote if available, for straight quotes in verbatim environments
\IfFileExists{upquote.sty}{\usepackage{upquote}}{}
% use microtype if available
\IfFileExists{microtype.sty}{%
\usepackage[]{microtype}
\UseMicrotypeSet[protrusion]{basicmath} % disable protrusion for tt fonts
}{}
\IfFileExists{parskip.sty}{%
\usepackage{parskip}
}{% else
\setlength{\parindent}{0pt}
\setlength{\parskip}{6pt plus 2pt minus 1pt}
}
\usepackage{hyperref}
\hypersetup{
            pdfborder={0 0 0},
            breaklinks=true}
\urlstyle{same}  % don't use monospace font for urls
\usepackage{longtable,booktabs}
% Fix footnotes in tables (requires footnote package)
\IfFileExists{footnote.sty}{\usepackage{footnote}\makesavenoteenv{longtable}}{}
\setlength{\emergencystretch}{3em}  % prevent overfull lines
\providecommand{\tightlist}{%
  \setlength{\itemsep}{0pt}\setlength{\parskip}{0pt}}
\setcounter{secnumdepth}{0}
% Redefines (sub)paragraphs to behave more like sections
\ifx\paragraph\undefined\else
\let\oldparagraph\paragraph
\renewcommand{\paragraph}[1]{\oldparagraph{#1}\mbox{}}
\fi
\ifx\subparagraph\undefined\else
\let\oldsubparagraph\subparagraph
\renewcommand{\subparagraph}[1]{\oldsubparagraph{#1}\mbox{}}
\fi

% set default figure placement to htbp
\makeatletter
\def\fps@figure{htbp}
\makeatother


\date{}

\begin{document}

\hypertarget{mac0422---ep3}{%
\section{MAC0422 - EP4}\label{mac0422---ep3}}

\hypertarget{integrantes}{%
\subsection{Integrantes}\label{integrantes}}

\title{\textbf{MAC0422 - EP4}}
\author{Daniel Martinez - 10297709 
  \and Pedro Paulo Bambace - 10297668}

\maketitle

\hypertarget{arquivos-criadosmodificados}{%
\subsection{Arquivos
criados/modificados}\label{arquivos-criadosmodificados}}

\begin{itemize}
\tightlist
\item
  \texttt{/usr/src/include/minix/const.h}
\item
  \texttt{/usr/include/minix/const.h}
\item
  \texttt{/usr/src/include/fcntl.h}
\item
  \texttt{/usr/include/fcntl.h}
\item
  \texttt{/usr/src/lib/posix/\_open\_tmp.c}
\item
  \texttt{/usr/src/lib/syscall/\_open\_tmp.s}
\item
  \texttt{/usr/src/lib/syscall/Makefile.in}
\item
  \texttt{/usr/src/lib/posix/Makefile.in}
\item
  \texttt{/usr/include/minix/callnr.h}
\item
  \texttt{/usr/src/include/minix/callnr.h}
\item
  \texttt{/usr/src/servers/fs/open.c}
\item
  \texttt{/usr/src/servers/fs/proto.h}
\item
  \texttt{/usr/src/servers/fs/table.c}
\item
  \texttt{/usr/src/servers/fs/file.h}
\item
  \texttt{/usr/src/servers/fs/path.c}
\item
  \texttt{/usr/src/servers/fs/proto.h}
\item
  \texttt{/root/banana.c}
\item
  \texttt{/root/uva}
\end{itemize}

\hypertarget{detalhes-de-implementauxe7uxe3o}{%
\subsection{Detalhes de
implementação}\label{detalhes-de-implementauxe7uxe3o}}

\hypertarget{implementauxe7uxe3o-do-open_tmp}{%
\subsubsection{\texorpdfstring{Implementação do
\texttt{open\_tmp()}}{Implementação do open\_tmp()}}\label{implementauxe7uxe3o-do-open_tmp}}

Primeiramente, definimos nosso valor de I\_TEMPORARY como 0030000, pois
tivemos dificuldade de fazer o número sugerido pelo enunciado se
comportar da maneira correta. Em seguida, criamos a chamada de
biblioteca \texttt{open\_tmp()} que realiza a chamada de sistema OPENTMP
no \emph{File System}. Essa chamada de sistema cria um novo arquivo
temporário que é deletado com o final da execução do programa. Para
isso, criamos em \texttt{open.c} uma cópia modificada da função
\texttt{common\_open()} (chamada \texttt{common\_open\_temp()})que salva
os \emph{i-nodes} com esse novo modo (I\_TEMPORARY), além de salvar o
\emph{i-node} do diretório desse arquivo em um novo campo da tabela
\texttt{filp}. Modificamos também a função \texttt{do\_close()} para que
quando o \emph{i-node} seja de um arquivo temporário e não houverem mais
nenhuma referência a ele, ele seja deletado. Para isso, criamos uma
cópia também da função \texttt{search\_dir()} (chamada
\texttt{search\_dir\_inode()}) que busca (e no nosso caso também deleta)
os filhos de um diretório baseado apenas em seu \emph{i-node}.

\hypertarget{build-e-testes}{%
\subsubsection{Build e testes}\label{build-e-testes}}

O mesmo arquivo de build dos EPs anteriores foi utilizado. Basta
executar ./b para compilar as mudanças do sistema.

\end{document}
