\PassOptionsToPackage{unicode=true}{hyperref} % options for packages loaded elsewhere
\PassOptionsToPackage{hyphens}{url}
%
\documentclass[]{article}
\usepackage{lmodern}
\usepackage{amssymb,amsmath}
\usepackage{ifxetex,ifluatex}
\usepackage{fixltx2e} % provides \textsubscript
\ifnum 0\ifxetex 1\fi\ifluatex 1\fi=0 % if pdftex
  \usepackage[T1]{fontenc}
  \usepackage[utf8]{inputenc}
  \usepackage{textcomp} % provides euro and other symbols
\else % if luatex or xelatex
  \usepackage{unicode-math}
  \defaultfontfeatures{Ligatures=TeX,Scale=MatchLowercase}
\fi
% use upquote if available, for straight quotes in verbatim environments
\IfFileExists{upquote.sty}{\usepackage{upquote}}{}
% use microtype if available
\IfFileExists{microtype.sty}{%
\usepackage[]{microtype}
\UseMicrotypeSet[protrusion]{basicmath} % disable protrusion for tt fonts
}{}
\IfFileExists{parskip.sty}{%
\usepackage{parskip}
}{% else
\setlength{\parindent}{0pt}
\setlength{\parskip}{6pt plus 2pt minus 1pt}
}
\usepackage{hyperref}
\hypersetup{
            pdfborder={0 0 0},
            breaklinks=true}
\urlstyle{same}  % don't use monospace font for urls
\usepackage{longtable,booktabs}
% Fix footnotes in tables (requires footnote package)
\IfFileExists{footnote.sty}{\usepackage{footnote}\makesavenoteenv{longtable}}{}
\setlength{\emergencystretch}{3em}  % prevent overfull lines
\providecommand{\tightlist}{%
  \setlength{\itemsep}{0pt}\setlength{\parskip}{0pt}}
\setcounter{secnumdepth}{0}
% Redefines (sub)paragraphs to behave more like sections
\ifx\paragraph\undefined\else
\let\oldparagraph\paragraph
\renewcommand{\paragraph}[1]{\oldparagraph{#1}\mbox{}}
\fi
\ifx\subparagraph\undefined\else
\let\oldsubparagraph\subparagraph
\renewcommand{\subparagraph}[1]{\oldsubparagraph{#1}\mbox{}}
\fi

% set default figure placement to htbp
\makeatletter
\def\fps@figure{htbp}
\makeatother


\date{}

\begin{document}

\title{\textbf{MAC0422 - EP3}}
\author{Daniel Martinez - 10297709 
  \and Pedro Paulo Bambace - 10297668}

\maketitle

\hypertarget{arquivos-criadosmodificados}{%
\subsection{Arquivos
criados/modificados}\label{arquivos-criadosmodificados}}

\begin{itemize}
\tightlist
\item
  \texttt{/usr/src/servers/pm/alloc.c}
\item
  \texttt{/root/memstat.c}
\item
  \texttt{/root/memstat}
\item
  \texttt{/root/makefile}
\end{itemize}

Além disso, adicionamos a ferramenta de testes disponibilizada pelo
professor em \texttt{/root/memuse/}

\hypertarget{detalhes-de-implementauxe7uxe3o}{%
\subsection{Detalhes de
implementação}\label{detalhes-de-implementauxe7uxe3o}}

\hypertarget{algoritmo-de-alocauxe7uxe3o-worst-fit}{%
\subsubsection{\texorpdfstring{Algoritmo de alocação \emph{worst
fit}}{Algoritmo de alocação worst fit}}\label{algoritmo-de-alocauxe7uxe3o-worst-fit}}

Na função \texttt{alloc\_mem()} do Process Manager, substituimos o
algoritmo que escolhe o \emph{buraco} na memória onde um novo espaço
será alocado pelo \emph{worst fit}. Para isso, iteramos todos os
\emph{buracos} disponíveis e escolhemos o maior para alocação da
memoria.

\hypertarget{estatuxedsticas-sobre-buracos-na-memuxf3ria}{%
\subsubsection{\texorpdfstring{Estatísticas sobre \emph{buracos} na
memória}{Estatísticas sobre buracos na memória}}\label{estatuxedsticas-sobre-buracos-na-memuxf3ria}}

Criamos o programa especificado no enunciado que utiliza a chamada de
sistema \texttt{getsysinfo()} para coletar informações sobre os
\emph{buracos} presentes na memória, que nos retorna um vetor de
\emph{buracos}. Com essa informação, criamos um vetor com o tamanho dos
\emph{buracos} existentes e sua quantidade, para o cálculo da média,
desvio padrão e mediana.

\hypertarget{arquivo-de-build}{%
\subsubsection{Arquivo de build}\label{arquivo-de-build}}

Para compilar o executável do \texttt{memstat}, basta executar
\texttt{make} no diretório \texttt{/root/}, como exemplificado no
enunciado.

\end{document}
