\PassOptionsToPackage{unicode=true}{hyperref} % options for packages loaded elsewhere
\PassOptionsToPackage{hyphens}{url}
%
\documentclass[]{article}
\usepackage{lmodern}
\usepackage{amssymb,amsmath}
\usepackage{ifxetex,ifluatex}
\usepackage{fixltx2e} % provides \textsubscript
\ifnum 0\ifxetex 1\fi\ifluatex 1\fi=0 % if pdftex
  \usepackage[T1]{fontenc}
  \usepackage[utf8]{inputenc}
  \usepackage{textcomp} % provides euro and other symbols
\else % if luatex or xelatex
  \usepackage{unicode-math}
  \defaultfontfeatures{Ligatures=TeX,Scale=MatchLowercase}
\fi
% use upquote if available, for straight quotes in verbatim environments
\IfFileExists{upquote.sty}{\usepackage{upquote}}{}
% use microtype if available
\IfFileExists{microtype.sty}{%
\usepackage[]{microtype}
\UseMicrotypeSet[protrusion]{basicmath} % disable protrusion for tt fonts
}{}
\IfFileExists{parskip.sty}{%
\usepackage{parskip}
}{% else
\setlength{\parindent}{0pt}
\setlength{\parskip}{6pt plus 2pt minus 1pt}
}
\usepackage{hyperref}
\hypersetup{
            pdfborder={0 0 0},
            breaklinks=true}
\urlstyle{same}  % don't use monospace font for urls
\usepackage{longtable,booktabs}
% Fix footnotes in tables (requires footnote package)
\IfFileExists{footnote.sty}{\usepackage{footnote}\makesavenoteenv{longtable}}{}
\setlength{\emergencystretch}{3em}  % prevent overfull lines
\providecommand{\tightlist}{%
  \setlength{\itemsep}{0pt}\setlength{\parskip}{0pt}}
\setcounter{secnumdepth}{0}
% Redefines (sub)paragraphs to behave more like sections
\ifx\paragraph\undefined\else
\let\oldparagraph\paragraph
\renewcommand{\paragraph}[1]{\oldparagraph{#1}\mbox{}}
\fi
\ifx\subparagraph\undefined\else
\let\oldsubparagraph\subparagraph
\renewcommand{\subparagraph}[1]{\oldsubparagraph{#1}\mbox{}}
\fi

% set default figure placement to htbp
\makeatletter
\def\fps@figure{htbp}
\makeatother

\title{\textbf{MAC0422 - EP2}}
\author{Daniel Martinez - 10297709 
  \and Pedro Paulo Bambace - 10297668}

\date{}

\begin{document}
\maketitle

\hypertarget{arquivos-criadosmodificados}{%
\subsection{Arquivos
criados/modificados}\label{arquivos-criadosmodificados}}

\hypertarget{visualizauxe7uxe3o-da-tabela-de-processos}{%
\paragraph{Visualização da tabela de
processos}}

\begin{itemize}
\tightlist
\item
  \texttt{minix\_os/usr/src/servers/is/dmp\_kernel.c}
\item
  \texttt{minix\_os/usr/src/servers/is/dmp.c}
\item
  \texttt{minix\_os/usr/src/servers/is/proto.h}
\item
  \texttt{minix\_os/usr/include/minix/callnr.h}
\item
  \texttt{minix\_os/usr/src/include/minix/callnr.h}
\end{itemize}

\hypertarget{chamada-de-sistema-para-alterauxe7uxe3o-de-prioridade}{%
\paragraph{Chamada de sistema para alteração de
prioridade}}

\begin{itemize}
\tightlist
\item
  \texttt{minix\_os/usr/include/sys/resource.h}
\item
  \texttt{minix\_os/usr/src/lib/posix/priority.c}
\item
  \texttt{minix\_os/usr/src/servers/pm/misc.c}
\item
  \texttt{minix\_os/usr/src/servers/pm/proto.h}
\item
  \texttt{minix\_os/usr/src/servers/pm/table.c}
\item
  \texttt{minix\_os/usr/src/lib/syslib/sys\_nice.c}
\item
  \texttt{minix\_os/usr/src/include/minix/syslib.h}
\item
  \texttt{minix\_os/usr/include/minix/syslib.h}
\item
  \texttt{minix\_os/usr/include/minix/com.h}
\item
  \texttt{minix\_os/usr/src/include/minix/com.h}
\item
  \texttt{minix\_os/usr/src/kernel/system.c}
\item
  \texttt{minix\_os/usr/src/kernel/system.h}
\item
  \texttt{minix\_os/usr/src/kernel/kernel.h}
\item
  \texttt{minix\_os/usr/src/kernel/system/Makefile}
\item
  \texttt{minix\_os/usr/src/kernel/system/.depend}
\item
  \texttt{minix\_os/usr/src/kernel/config.h}
\item
  \texttt{minix\_os/usr/src/kernel/system/do\_priority.c}
\end{itemize}

\hypertarget{outros}{%
\paragraph{Outros}}

\begin{itemize}
\tightlist
\item
  \texttt{minix\_os/usr/src/b} - Arquivo para automatizar o build
\item
  \texttt{minix\_os/root/teste.c} - Arquivo que testa a chamada
  implementada
\end{itemize}

\hypertarget{detalhes-de-implementauxe7uxe3o}{%
\subsection{Detalhes de
implementação}\label{detalhes-de-implementauxe7uxe3o}}

\hypertarget{visualizauxe7uxe3o-da-tabela-de-processos-1}{%
\subsubsection{Visualização da tabela de
processos}\label{visualizauxe7uxe3o-da-tabela-de-processos-1}}

Nessa parte do EP, pegamos a tabela de processos através da chamada de
sistema \texttt{sys\_getproctab()} e a ordenamos em ordem decrescente de
prioridade (na verdade, ordenamos em ordem crescente de p\_priority, que
quanto menor é, maior a prioridade do processo). Definimos que quando o
usuário pressiona a tecla F4, imprimimos para cada processo dessa tabela
as seguintes informações:

\begin{itemize}
\tightlist
\item
  nr do processo
\item
  Nome do processo
\item
  Prioridade de execução
\item
  PID do processo
\item
  Tempo de CPU
\item
  Tempo de Sistema
\item
  Endereço do ponteiro da pilha
\end{itemize}

Para conseguir o PID do processo, foi implementada uma chamada de
biblioteca \texttt{getpidfromnr()} que retorna o PID do processo dado
seu nr. Ela executa a chamada de sistema \texttt{CHPRIORITY} do
\emph{Process Manager} como modo 1, que relaciona as tabelas de
processos \texttt{mproc} e \texttt{proc} e retorna o PID do processo
especificado.

Se houver mais processos que os que cabem na tela, eles serão exibidos
na próxima vez que o usuário pressionar F4.

\hypertarget{chamada-de-sistema-para-alterauxe7uxe3o-de-prioridade-1}{%
\subsubsection{Chamada de sistema para alteração de
prioridade}\label{chamada-de-sistema-para-alterauxe7uxe3o-de-prioridade-1}}

Implementamos a chamada de biblioteca \texttt{chpriority()} como
especificado no enunciado, que executa a mesma chamada de sistema
\texttt{CHPRIORITY}, mas como modo 0. Esta verifica se a prioridade
passada está entre \texttt{MAX\_USER\_Q} e \texttt{MIN\_USER\_Q}, que
são as prioridades permitidas para um processo de usuário e também se o
processo cujo PID foi passado é filho do processo que a chamou através
do atributo \texttt{mp\_parent} do processo. Em seguida, ela executa a
chamada de \emph{kernel} \texttt{sys\_priority} para a alteração da
prioridade. Essa chamada de \emph{kernel} retira o processo da fila de
execução, altera sua prioridade e prioridade máxima, e o coloca na nova
fila.

Para retornar números negativos nem a utilização de \texttt{errno},
utilizamos a função \texttt{\_taskcall()} no lugar da
\texttt{\_syscall()}.

\hypertarget{arquivo-de-teste}{%
\subsubsection{Arquivo de teste}\label{arquivo-de-teste}}

Há um arquivo de teste em \texttt{/root/teste.c} que testa a chamada de
sistema implementada nesse EP para três casos:

\begin{itemize}
\tightlist
\item
  Processo não-filho, onde o resultado esperado é -2
\item
  Prioridade inválida, onde o resultado esperado é -1
\item
  Prioridade válida para um processo filho, onde o resultado esperado é
  a mudança de prioridade de tal processo, retornando seu valor.
\end{itemize}

Nesse último caso, utilizamos a chamada de sistema \texttt{fork()} para
a criação de um processo filho, que fica somente espera 100 segundos e
termina, enquanto o processo pai faz a chamada de sistema implementada,
também esperando 100 segundos depois disso para que seja possível
verificar se o valor realmente foi alterado na tabela de processos
(pressionando F4).

\hypertarget{arquivo-de-build}{%
\subsubsection{Arquivo de build}\label{arquivo-de-build}}

É um simples script que automatiza o processo de build. Ele executa o
\texttt{make\ world} para compilar todas as partes do Minix, move a
imagem de \emph{kernel} gerada para \texttt{/boot/image/image\_big} e se
tudo der certo, reinicia o sistema automaticamente.

\end{document}
